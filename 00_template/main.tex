\documentclass[
    parskip=half, 
    twoside=false,
    twocolumn=true
]{scrarticle}
\input{../preamble.tex}

\begin{document}

\title{title}
\subtitle{subtitle}
\author{Aurel Müller-Schoenau, Leon Oleschko}
\date{\dotdate\today}


% make a custom title page
\begin{titlepage}
    \sffamily

    \vspace*{3cm}
    {
        \fontsize{32}{32}
        \markieren{}{}{Raman}{Spektroskopie}
    }
    \vspace{.25cm}\\
    {
        \Large
        Aurel Müller-Schoenau, Leon Oleschko\\
        Supervised by Aasdf Kasdf
        \vspace{.05cm}\\
        10. Januar 2023
        \vspace{.25cm}\\
        \normalsize
        Physikalisches Fortgeschrittenenpraktikum 2\\
        Universität Konstanz
    }
    \vspace{3cm}\\
    {
        Abstract auf Englisch (10-15 Zeilen)
        \blindtext[2]
    }
\end{titlepage}


\section{Introduction}
\blindtext

\subsection{Physical Principles}
kompakten Zusammenstellung der physikalischen Grundlagen
\blindtext

\section{Methods}
Mit einer Skizze des Versuchsaufbaus
\blindtext[3]

\pagebreak
\section{Procedure}
\blindtext[5]

\pagebreak
\section{Results}
\begin{figure}[H]
    \centering
    \includegraphics{fig/2024-10-01 line plot.pdf}
    \caption{Inline plot}
\end{figure}
\begin{figure*}
    \centering \includegraphics{fig/2024-10-01 img plot.pdf}
    \caption{This is a figure caption}
\end{figure*}
\blindtext[3]

\subsection{Error Analysis}
\blindtext

All recorded data and the analysis is available at \url{www.github.com/leoole100/fp2}.

\pagebreak
\section{Discussion}
\blindtext


\end{document}