\documentclass[
    parskip=half, 
    twoside=false,
    twocolumn=true,
    fontsize=11pt,
]{scrarticle}
\input{../preamble.tex}
\addbibresource{literature.bib}

\begin{document}

\title{title}
\subtitle{subtitle}
\author{Aurel Müller-Schoenau, Leon Oleschko}
\date{\dotdate\today}


% make a custom title page
\begin{titlepage}
    \sffamily
    \vspace*{3cm}
    {
        \fontsize{32}{32}
        \markieren{}{}{}{Rauschen}
    }
    \vspace{.25cm}\\
    {
        \Large
        Aurel Müller-Schoenau, Leon Oleschko\\
        Supervised by Richard Schlitz
        \vspace{.05cm}\\
        27. November 2024
        \vspace{.25cm}\\
        \normalsize
        Physikalisches Fortgeschrittenenpraktikum 2\\
        Universität Konstanz
    }
    \vfill
    {
        \normalfont\normalsize
        
    }
    \vfill
    \begin{flushright}
        Available at \url{www.github.com/leoole100/fp2}.
    \end{flushright}
\end{titlepage}

\section{Introduction}

All measurements conducted in the real world are subject to some form of noise. What is often regarded an obstacle may provide some important information about the system that is being observed. In this experiment, we take a look at two types of noise, namely Johnson noise occuring in an electrical resistor and Shot noise of a photo diode. Some of the properties of these types of noise are universal and will allow us to derive the Boltzmann constant and the elementary charge as well as the point of absolute zero temperature.

\section{Setup}
For this experiment, the setup \textit{Noise Fundamentals NF1-A} by \textit{TeachSpin, Inc.} was used. It consists of a low level electronics (LLE) component with a built-in preamplifier, and in a separate enclosure the High Level Electronics such as frequency filters and an additional output amplifier. The resistors used in the first part of the experiment are built directly into the LLE setup, whereas the module carrying the photo diode and lamp to measure shot noise has to be connected separately. An additional probe with resistors sits inside a Dewar which can hold liquid Nitrogen. Since we are measuring wide-band noise, the measurement bandwidth is important. The effective bandwidth resulting from the settings in the HLE component is taken from a table provided in the setup manual \autocite{instructions}.\\
Measurements are taken with a multimeter connected to the output of the HLE. The statistical uncertainty of the voltmeter was measured to be $\sigma = \SI{0.001}{V}$.

\subsection*{Oscilloscope RMS Measurement}
An oscilloscope connected to the setup was supposed to be used for RMS measurements. That would have been favourible to understand the influence of different parameters at a glance. But during initial testing, we found that the readings on the device differed significantly from what the multimeter was showing, and were susceptible to axis scaling on in the oscillosope graph. A few unsuccessful attempts later, the device was ruled out of the experiment due to what we consider a firmware bug.\\

\section{Johnson-Nyquist Noise}
\begin{figure*}[h!]
    \centering
    \includegraphics{figures/01 johnson noise.pdf}
    \caption{
        Measured noise density over different resistors for different bandwidths.
        Linear fit residuals are shown in the lower plot.
        Uncertainties are estimated using the Uncertainties of the voltage measurements.
        The uncertainties of the resistors and the amplifiers are not considered.
    }
    \label{fig:johnson noise}
\end{figure*}

\begin{equation}
    S = 4\; k_B T\; R + S_0
\end{equation} 
Therefore $k_B T$ and the amplifier noise $S_0$ can be measured from the fit in \autoref{fig:johnson noise}.
Assuming $T=\SI{295(3)}{K}$, the measured value for $k_B$ is $\SI{1.416(19)e-23}{J/K}$, which is off the literature value of $\SI{1.380e-23}{J/K}$.
The amplifier noise is measured to be $\SI{6.168(55)e-17}{V^2/Hz}$.

\subsubsection*{Temperature Measurement}
\begin{figure*}[h!]
    \centering
    \includegraphics{figures/02 temperature.pdf}
    \caption{
        Measured noise density over temperature for different resistors and bandwidths.
    }
    \label{fig:johnson noise temperature}
\end{figure*}
\begin{figure}[h!]
    \centering
    \includegraphics{figures/02 temperature distribution.pdf}
    \caption{
        Distribution of the measured temperature by looking at the crossings of the linear fits in \autoref{fig:johnson noise temperature}.
    }
    \label{fig:johnson noise temperature distribution}
\end{figure}
The measured zero temperature is \SI{-6.4(15)}{K}, the uncertainty is only including the spread.

\section{Shot Noise}
\begin{figure}[h!]
    \centering
    \includegraphics{figures/03 shot noise.pdf}
    \caption{
        Distribution of the measured value for $e$, by measuring the voltage over a resistor (blue) and using a transimpedance amplifier (yellow).
    }
    \label{fig:shot noise}
\end{figure}
The estimated values for $e$ using the two methods respectively are $\SI{1.76(25)e-19}{C}$ and $\SI{1.9(82)e-19}{C}$, which are both in agreement with the literature value of $\SI{1.602e-19}{C}$.
\textbf{Why is the second Uncertainty so high? See Histogram.}

\pagebreak
\section{Discussion}

\addcontentsline{toc}{section}{Literature}
\nocite{*}
\printbibliography

\end{document}
