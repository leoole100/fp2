\documentclass[
    parskip=half, 
    twoside=false,
    twocolumn=true,
    fontsize=11pt,
]{scrarticle}
\input{../preamble.tex}
\addbibresource{literature.bib}

\begin{document}

\title{title}
\subtitle{subtitle}
\author{Aurel Müller-Schoenau, Leon Oleschko}
\date{\dotdate\today}


% make a custom title page
\begin{titlepage}
    \sffamily
    \vspace*{3cm}
    {
        \fontsize{32}{32}
        \markieren{}{}{}{Rauschen}
    }
    \vspace{.25cm}\\
    {
        \Large
        Aurel Müller-Schoenau, Leon Oleschko\\
        Supervised by Richard Schlitz
        \vspace{.05cm}\\
        27. November 2024
        \vspace{.25cm}\\
        \normalsize
        Physikalisches Fortgeschrittenenpraktikum 2\\
        Universität Konstanz
    }
    \vfill
    {
        \normalfont\normalsize
        
    }
    \vfill
    \begin{flushright}
        Available at \url{www.github.com/leoole100/fp2}.
    \end{flushright}
\end{titlepage}

\section{Introduction}

All measurements conducted in the real world are subject to some form of noise. What is often regarded an obstacle may provide some important information about the system that is being observed. In this experiment, we take a look at two types of noise, namely Johnson-Nyquist noise occuring in an electrical resistor and Shot noise of a photo diode. Some of the properties of these types of noise are universal and will allow us to derive the Boltzmann constant and the elementary charge as well as the point of absolute zero temperature.

\section{Setup}
For this experiment, the setup \textit{Noise Fundamentals NF1-A} by \textit{TeachSpin, Inc.} was used. It consists of a low level electronics (LLE) component with a built-in preamplifier, and in a separate enclosure the High Level Electronics such as frequency filters and an additional output amplifier. The resistors used in the first part of the experiment are built directly into the LLE setup, whereas the module carrying the photo diode and lamp to measure shot noise has to be connected separately. An additional probe with resistors sits inside a Dewar which can hold liquid Nitrogen. Since we are measuring wide-band noise, the measurement bandwidth is important. The effective bandwidth resulting from the settings in the HLE component is taken from a table provided in the setup manual \autocite{instructions}.\\
Measurements are taken with a multimeter connected to the output of the HLE. The statistical uncertainty of the voltmeter was measured to be $\sigma = \SI{0.001}{V}$.

\subsection*{Oscilloscope RMS Measurement}
An oscilloscope connected to the setup was supposed to be used for RMS measurements. That would have been favourible to understand the influence of different parameters at a glance. But during initial testing, we found that the readings on the device differed significantly from what the multimeter was showing, and were susceptible to axis scaling on in the oscillosope graph. A few unsuccessful attempts later, the device was ruled out of the experiment due to what we consider a firmware bug.\\

\section{Johnson-Nyquist Noise}

\begin{figure*}[h!]
    \centering
    \includegraphics{figures/measure_R_setup_part1.pdf}
    \caption{
        Schematic for the setup to measure Johnson noise across an ohmic resistor $R_\text{IN}$ as provided by \autocite{instructions}.
    }
    \label{fig:johnson setup R}
\end{figure*}

\begin{figure*}[h!]
    \centering
    \includegraphics{figures/01 johnson noise.pdf}
    \caption{
        Measured noise density over different resistors for different bandwidths.
        Linear fit residuals are shown in the lower plot.
        Uncertainties are estimated using the Uncertainties of the voltage measurements.
        The uncertainties of the resistors and the amplifiers are not considered.
    }
    \label{fig:johnson noise}
\end{figure*}

Conducting an experiment at room temperature means that a certain amount of thermal excitation is going to be present. The voltage across an electric component such as a simple ohmic resistor therefore shows small fluctuations. This so-called \textit{Johnson-Nyquist Noise} is white noise. Its power spectral density is quantified by the voltage variance

\begin{equation}
    \label{eq:johnson noise}
    \langle V_J^2 \rangle = 4\; k_B T\; R\; \Delta f + S_0
\end{equation} 

Where $k_B$ is the Boltzmann constant, $T$ is temperature and $R$ is the electric resistance of the resistor. The bandwidth $\Delta f$ is crucial since the data is measured digitally. It is adjusted using the high pass and low pass filters in the HLE component. The offset $S_0$ was added to the equation taken from \autoref{instructions}. It has its roots in the signal processing chain. Using the setup to probe a very small resistance should yield $S_0$, but it can also be eliminated from the equation by taking the derivate with respect to $R$, assuming that resistor and amplifier operate independently. The schematic for this measurement is shown in \autoref{fig:johnson setup R}. The resistor $R_\text{IN}$ is the one being probed. The preamplifier has a gain of $6$.  According to \autoref{eq:johnson noise}, the noise is linear in $R$. It was measured for multiple resistors and bandwidths. The results are shown in \autoref{fig:johnson noise}. Since the resistance and temperature are known, the Boltzmann constant can be determined from the linear fits shown in the graph. Assuming $T=\SI{295(3)}{K}$, the measured value for $k_B$ is $\SI{1.416(19)e-23}{J/K}$, which is off the literature value of $\SI{1.380e-23}{J/K}$. The offset noise generated by the signal chain is measured to be $\SI{6.168(55)e-17}{V^2/Hz}$.

\subsubsection*{Temperature Measurement}
\begin{figure*}[h!]
    \centering
    \includegraphics{figures/02 temperature.pdf}
    \caption{
        Measured noise density over temperature for different resistors and bandwidths.
    }
    \label{fig:johnson noise temperature}
\end{figure*}
\begin{figure}[h!]
    \centering
    \includegraphics{figures/02 temperature distribution.pdf}
    \caption{
        Distribution of the measured temperature by looking at the crossings of the linear fits in \autoref{fig:johnson noise temperature}.
    }
    \label{fig:johnson noise temperature distribution}
\end{figure}
In the previous section, the absolute room temperature was assumed to be already known. However, since \autoref{eq:johnson noise} is linear in $T$ (neglecting the offset noise $S_0$, which has already been quantified) extrapolating the measured data to the point where the noise vanishes should yield the value $T$ at absolute zero. To do this, a probe containing several resistors is lowered into a Dewar. The probe is connected to the LLE using an otherwise similar setup to \autoref{fig:johnson setup R}. The noise RMS was measured at room temperature and with the probe submerged in liquid Nitrogen. Again, the measurement was conducted for different bandwidths. The results are shown in \autoref{fig:johnson noise temperature} including linear fits.
The measured zero temperature is \SI{-6.4(15)}{K}. The measurement errors were so small that the uncertainty was calculated from only the spread of data points, but the result implies other error sources.

\section{Shot Noise}
\begin{figure}[h!]
    \centering
    \includegraphics{figures/03 shot noise.pdf}
    \caption{
        Distribution of the measured value for $e$, by measuring the voltage over a resistor (blue) and using a trans-impedance amplifier (yellow).
    }
    \label{fig:shot noise}8
\end{figure}
\begin{figure}[h!]
    \centering
    \includegraphics{figures/tamp_schematic.pdf}
    \caption{
        Current measurement across a photo diode using a transimpedance amplifier setup. Assuming no current flows into the amplifier input, the current through $R_f$ has to be exactly the current $I + I_0$ flowing through the diode. This allows for very precise measurement of $\langle I^2 \rangle$.
    }
    \label{fig:transimpedance schematic}
\end{figure}


Charge is transported through a conductor only in multiples of the elementary charge $e$. This produces a type of noise called \textit{Shot Noise} which becomes relevant for very small currents. The variance of the noise current across an electric component is
\begin{equation}
    \label{eq:shot noise}
    \langle I^2 \rangle = 2 e I_0 \Delta f + S_0
\end{equation}
as shown in \autocite{Buch}. Note that the total current is $I_0 + I$ with the mean noise current $\langle I \rangle = 0$. Again, an additional noise source $S_0$ was added, accounting for the signal processing chain. According to \autoref{eq:shot noise} the value of $e$ can be calculated if $I_0$ is known, because the additional component $S_0$ can be eliminated by measuring the same component at different currents, and computing the derivative
\begin{equation}
 e = \frac{1}{2}\frac{\Delta S_I}{\Delta I_0}
\end{equation}
In the second part of the experiment, shot noise was measured across a photo diode. The current $I_0$ can be measured in two different ways: It can be determined by connecting a resistor in series with the diode and measuring the voltage drop across it, similar to how Johnson noise was measured in the previous part. Another way is to use an amplifier which produces an output voltage proportional to an input current, a so-called \textit{transimpedance amplifier}. The schematic using the latter setup is shown in \autoref{fig:transimpedance schematic}. A lamp with variable brightness next to the photo diode allowed for adjustment of the mean current $I_0$.
The estimated values for $e$ using the two methods respectively are $\SI{1.76(25)e-19}{C}$ and $\SI{1.9(08)e-19}{C}$, which are both in agreement with the literature value of $\SI{1.602e-19}{C}$.

\pagebreak
\section{Discussion}

\addcontentsline{toc}{section}{Literature}
\nocite{*}
\printbibliography

\end{document}
