\documentclass[
    parskip=half, 
    twoside=false,
    twocolumn=true,
    fontsize=11pt,
]{scrarticle}
\input{../preamble.tex}

\begin{document}

\title{title}
\subtitle{subtitle}
\author{Aurel Müller-Schoenau, Leon Oleschko}
\date{\dotdate\today}


% make a custom title page
\begin{titlepage}
    \sffamily
    \vspace*{3cm}
    {
        \fontsize{32}{32}
        \markieren{}{}{Evanescent light scattering}{Optical Tweezers}
    }
    \vspace{.25cm}\\
    {
        \Large
        Aurel Müller-Schoenau, Leon Oleschko\\
        Supervised by Aasdf Kasdf
        \vspace{.05cm}\\
        13.11.2024
        \vspace{.25cm}\\
        \normalsize
        Physikalisches Fortgeschrittenenpraktikum 2\\
        Universität Konstanz
    }
    \vfill
    {
        \normalfont\normalsize
        Abstract auf Englisch (10-15 Zeilen)
        \blindtext[2]
    }
    \vfill
    \begin{flushright}
        Available at \url{www.github.com/leoole100/fp2}.
    \end{flushright}
\end{titlepage}

\section{Introduction}

\subsection{Physical Principles}
kompakten Zusammenstellung der physikalischen Grundlagen

Mean square deviation and velocity autocorrelation:
\url{https://de.wikipedia.org/wiki/Mittlere_quadratische_Verschiebung#Verbindung_zur_Geschwindigkeitsautokorrelation}
\begin{align}
    &&\left<v(t)\cdot v(t+\tau)\right> &= - \frac{d}{d\tau} \frac{\left<r^2(\tau)\right>}{6\tau}\\
    \Rightarrow&&\left<r^2(\tau)\right> &= 6 \int_0^\tau (\tau - s) \left<v(0) \cdot v(s)\right> ds
\end{align}

The maxwell boltzmann relations (from script) are given by
\begin{align}
    P(x) &\propto \exp\left(- \frac{V(x)}{k_B T}\right)\\
    V(x) &\propto - \frac{\log{P(x)}}{k_B T}
\end{align}


\section{Methods}
Mit einer Skizze des Versuchsaufbaus

\section{Procedure}

\pagebreak
\section{Results}

All recorded data and the analysis is available at \url{www.github.com/leoole100/fp2}.

\pagebreak
\section{Discussion}


\end{document}