\documentclass[
    parskip=half, 
    twoside=false,
    twocolumn=true,
    fontsize=11pt,
]{scrarticle}
\input{../preamble.tex}
\addbibresource{literature.bib}

\begin{document}

\title{title}
\subtitle{subtitle}
\author{Aurel Müller-Schoenau, Leon Oleschko}
\date{\dotdate\today}


% make a custom title page
\begin{titlepage}
    \sffamily
    \vspace*{3cm}
    {
        \fontsize{32}{32}
        \markieren{}{}{Evanescent light scattering}{Optical Tweezers}
    }
    \vspace{.25cm}\\
    {
        \Large
        Aurel Müller-Schoenau, Leon Oleschko\\
        Supervised by Krishna Kumar, Karthika
        \vspace{.05cm}\\
        13.11.2024
        \vspace{.25cm}\\
        \normalsize
        Physikalisches Fortgeschrittenenpraktikum 2\\
        Universität Konstanz
    }
    \vfill
    {
        \normalfont\normalsize
        Abstract auf Englisch (10-15 Zeilen)
        \blindtext[2]
    }
    \vfill
    \begin{flushright}
        Available at \url{www.github.com/leoole100/fp2}.
    \end{flushright}
\end{titlepage}

\section{Introduction}

\subsection{Physical Principles}
kompakten Zusammenstellung der physikalischen Grundlagen


\section{Methods}
Mit einer Skizze des Versuchsaufbaus

\section{Procedure}

\section{Results}
All recorded data and the analysis is available at \url{www.github.com/leoole100/fp2}.

\pagebreak
\subsection{Transmission Light Microscopy}
\begin{figure}
    \centering
    \includegraphics{figures/01_01_1_particle.pdf}
    \caption{Transmission light microscopy of a observed particle. The radius of the particle is \SI{14(2)}{px}, equivalent to \SI{1.86(27)}{\micro m}.}
    \label{fig:01particle}
\end{figure}
In the first section of the experiment, the particles were observed using a transmission light microscope setup.
For this images with a resolution of $600\times 800$\si{px} were recorded with a frequency of \SI{10}{Hz} for \SI{10}{min}.
The magnification of the microscope was assumed to be \SI{0.13319672}{\micro m/px} \cite{instructions}, this is the main systematic error of this measurement procedure.

The images were normalized with a black (illumination off) and white (illumination on, particle not in frame) reference image, to remove the influence of dust in the imaging elements.
The particle that was used for this experiment is shown in \autoref{fig:01particle}, after the normalization.

\begin{figure*}
    \centering
    \includegraphics{figures/01_03_1_bivariate.pdf}
    \caption{Density of recorded particle positions, grouped by optical trap stiffness. The red circle indicates the approximate radius of the optical trap \SI{2.5}{\micro m}.}
    \label{fig:01bivariate}
\end{figure*}
To determine the trajectory of the particle, a effective center of mass was calculated for each frame:
\begin{equation}
    \vec{r}(t) = \iint \vec{r} \cdot \left(1-I(\vec{r}, t)\right)^2 d\vec{r}    
\end{equation}
The density of the resulting trajectory is shown in \autoref{fig:01bivariate} for different optical trap stiffnesses.
The approximate radius of the optical trap of \SI{2.5}{\micro m} is drawn as a red circle.
For the trap stiffness of \SI{0}{}, the particle is free to wander around, for the higher stiffnesses like \SI{1.01}{} the particle is mostly confined to the trap.\\
For a weak trap like \SI{0.75}{} the particle is still mostly confined to the trap, but can escape the linear trap region and randomly wander around. 
This happened multiple times during the shown measurement.

\begin{figure}
    \centering
    \includegraphics{figures/01_02_2_msd.pdf}
    \caption{Mean Square Displacement for different optical trap stiffnesses, fit: \autoref{eq:01_mdl_msd}}
    \label{fig:01msd}
\end{figure}
\subsection*{Mean Square Displacement}
A method to describe the trajectory of a random walk is the mean square displacement (MSD) \cite{wiki:msd}.
This is defined as the average of the squared distance of the particle from the starting point:
\begin{equation}
    \text{MSD}(t) = \left<\Delta r^2(t)\right> = \frac{1}{N} \sum_i^N \left(x_i(t) - x_i(0) \right)^2 
\end{equation}
Here a different implementation using the autocorrelation of the velocity was used, to achieve a more stable result.
This was implemented by \cite{jl:msd}.
The resulting MSD for different optical trap stiffnesses is shown in \autoref{fig:01msd}.

The MSD can be described by the following model:
\begin{equation}
    \text{MSD}(\tau) = \frac{1}{\frac{1}{D_0 \tau} + \frac{1}{\text{MSD}(\infty)}}
    \label{eq:01_mdl_msd} 
\end{equation}

For a free particle the $\text{MSD}(\infty)=\infty$ and the MSD grows linearly with $D_0$ over time \cite{wiki:msd,instructions}.\\
The linear fit for the free particles, works as described with a $D_0 = \SI{0.9357(25)}{\micro m^2/s}$.
This is roughly equivalent with the expected value of $D_0 = \frac{k_B T}{6 \pi \eta r} = \SI{1.22(18)}{\micro m ^2 / s}$ \cite{instructions}, with $r = \SI{1.86(27)}{\micro m}$ and $\eta = \SI{0.955}{mPa\cdot s}$.

For a confined particle the MSD reaches a plateau at $\text{MSD}(\infty)$ \cite{instructions}.
For the higher spring stiffnesses (\SI{0.90}{}, \SI{1.01}{}), the MSD reaches a plateau and the spring stiffness are shown in \autoref{fig:01spring}.\\
For the lower measured spring stiffness (\SI{0.75}{}), the MSD does not reach a plateau, as it partially escapes the trap and wanders around (see \autoref{fig:01bivariate}).
Therefore this procedure is not adequate for such low spring stiffnesses.

\begin{figure}
    \centering
    \includegraphics{figures/01_03_3_axis.pdf}
    \caption{Grouped by axis, relative to mean.}
\end{figure}
\subsubsection*{Potential}
The maxwell boltzmann relations \cite{instructions} are given by
\begin{align}
    P(x) &= \exp\left(- \frac{V(x)}{k_B T}\right)
    \Rightarrow V(x) = - \frac{\log{P(x)}}{k_B T}
\end{align}

\begin{figure}
    \centering
    \includegraphics{figures/01_03_4_spring_constants.pdf}
    \caption{Differently measured spring constants.}
    \label{fig:01spring}
\end{figure}

\clearpage
\subsection{Total Internal Reflection Microscopy}

\clearpage
\section{Discussion}
\blindtext

\addcontentsline{toc}{section}{Literature}
\nocite{*}
\printbibliography

\end{document}