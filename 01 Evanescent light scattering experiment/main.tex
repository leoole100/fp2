\documentclass[
    parskip=half, 
    twoside=false,
    twocolumn=true,
    fontsize=11pt,
]{scrarticle}
\input{../preamble.tex}

\begin{document}

\title{title}
\subtitle{subtitle}
\author{Aurel Müller-Schoenau, Leon Oleschko}
\date{\dotdate\today}


% make a custom title page
\begin{titlepage}
    \sffamily
    \vspace*{3cm}
    {
        \fontsize{32}{32}
        \markieren{}{}{Evanescent light scattering}{Optical Tweezers}
    }
    \vspace{.25cm}\\
    {
        \Large
        Aurel Müller-Schoenau, Leon Oleschko\\
        Supervised by Aasdf Kasdf
        \vspace{.05cm}\\
        13.11.2024
        \vspace{.25cm}\\
        \normalsize
        Physikalisches Fortgeschrittenenpraktikum 2\\
        Universität Konstanz
    }
    \vfill
    {
        \normalfont\normalsize
        Abstract auf Englisch (10-15 Zeilen)
        \blindtext[2]
    }
    \vfill
    \begin{flushright}
        Available at \url{www.github.com/leoole100/fp2}.
    \end{flushright}
\end{titlepage}

\section{Introduction}

\subsection{Physical Principles}
kompakten Zusammenstellung der physikalischen Grundlagen

Mean square deviation and velocity autocorrelation:
\url{https://de.wikipedia.org/wiki/Mittlere_quadratische_Verschiebung#Verbindung_zur_Geschwindigkeitsautokorrelation}
\begin{align}
    &&\left<v(t)\cdot v(t+\tau)\right> &= - \frac{d}{d\tau} \frac{\left<r^2(\tau)\right>}{6\tau}\\
    \Rightarrow&&\left<r^2(\tau)\right> &= 6 \int_0^\tau (\tau - s) \left<v(0) \cdot v(s)\right> ds
\end{align}

The maxwell boltzmann relations (from script) are given by
\begin{align}
    P(x) &\propto \exp\left(- \frac{V(x)}{k_B T}\right)\\
    V(x) &\propto - \frac{\log{P(x)}}{k_B T}
\end{align}


\section{Methods}
Mit einer Skizze des Versuchsaufbaus

\section{Procedure}

\pagebreak
\section{Results}
All recorded data and the analysis is available at \url{www.github.com/leoole100/fp2}.

\subsection{Transmission Light Microscopy}
The recorded images have a shape of $600\times 800$\si{px}.
The observed particle has a radius of \SI{14(2)}{px} equivalent to \SI{1.86(27)}{\micro m}. 
\begin{figure*}[h]
    \centering
    \includegraphics{figures/01_01_1_particle.pdf}
    \caption{Transmission light microscopy of a single particle, after normalization.}
\end{figure*}

\begin{figure*}[h]
    \centering
    \includegraphics{figures/01_02_1_trajectories.pdf}
    \caption{Tracked trajectories relative to trap center.}
\end{figure*}
\begin{figure*}[h]
    \centering
    \includegraphics{figures/01_02_2_center_distances.pdf}
    \caption{Distance to center of trap over time.}
\end{figure*}

Model for the MSD with $k=\alpha P \cdot\nicefrac{k_B T}{\text{MSD}(\infty)}$ with the optical tweezer power $P$ in arbitrary units:
\begin{align}
    \frac{1}{\text{MSD}(\tau, k)}
    &= \frac{1}{D_0 \tau} + \frac{1}{\text{MSD}(\infty)}\\
    &= \frac{1}{D_0 \tau} + \frac{1}{(k_B T \cdot P)_i}
    \label{eq:01_mdl_msd} 
\end{align}
For the $i$-th measurement.

\begin{figure*}[h]
    \centering
    \includegraphics{figures/01_02_2_msd.pdf}
    \caption{Mean Square Displacement, Linear Drift removed, fit: \autoref{eq:01_mdl_msd}}
\end{figure*}

\begin{figure*}[h]
    \centering
    \includegraphics{figures/01_03_1_bivariate.pdf}
    \caption{Bivariate histogram for different trap strengths, relative to mean position, different scale in um.}
\end{figure*}

\begin{figure*}[h]
    \centering
    \includegraphics{figures/01_03_2_radial.pdf}
    \caption{Distance to center of trap. Qualitatively, the same if center is mean per track or "real" center.}
\end{figure*}

\begin{figure*}[h]
    \centering
    \includegraphics{figures/01_03_3_axis.pdf}
    \caption{Grouped by axis, relative to mean.}
\end{figure*}

\begin{figure*}[h]
    \centering
    \includegraphics{figures/01_03_4_spring_constants.pdf}
    \caption{Differently measured spring constants.}
\end{figure*}

\clearpage
\subsection{Total Internal Reflection Microscopy}
\begin{align}
    P(\Delta z) &= N(0, \sigma^2) = N\left(0, \frac{\Delta T}{\left<T_s\right>} \sigma_s^2\right)\\
    \Delta I &= - I_0 \beta \exp\left(-\frac{z}{\beta}\right) \Delta z \\
    P(\Delta I) &= P(\Delta z) \left|\frac{d \Delta z}{d \Delta I}\right|\\
    &= N(0, \sigma^2) \;\frac{1}{I_0\beta} \exp\left(\frac{z}{\beta}\right)
\end{align}

\clearpage
\section{Discussion}


\end{document}